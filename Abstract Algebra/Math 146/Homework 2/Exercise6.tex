\subsection*{Exercise 22}
We want to know for which values of $a = 1,2,3,4$, $\mathbb{Z}_5[x]/\langle x^2 + a \rangle$ is a field. Given a field $F$ and polynomial $f(x) \in F[x]$, $F[x]/\langle f(x) \rangle$ is a field if and only if $f(x)$ is irreducible over $F$. Since $f(x) = x^2 + a$ is a polynomial of degree 2, we need only check whether or not $f(x)$ has roots in $\mathbb{Z}_5$ for each value of $a$: \\

\textbf{[i]} \underline{For $a = 1$, $f(x) = x^2 + 1$:}

$f(0) = (0 + 1) = 1 \not\in [0] \\
f(1) = (1 + 1) = 2 \not\in [0] \\
f(2) = (4 + 1) = 5 \in [0]$

Now, since 2 is a root of $f(x)$, $f(x)$ is reducible over $\mathbb{Z}_5$ and $\mathbb{Z}_5[x]/\langle x^2 + 1 \rangle$ is not a field. \\

\textbf{[ii]} \underline{For $a = 2, f(x) = x^2 + 2$:}

$f(0) = (0 + 2) = 2 \not\in [0] \\
f(1) = (1 + 2) = 3 \not\in [0] \\
f(2) = (4 + 2) = 6 \not\in [0] \\
f(3) = (9 + 2) = 11 \not\in [0] \\
f(4) = (16 + 2) = 18 \not\in [0]$

Now, since $f(x)$ has no roots in $\mathbb{Z}_5$, it is irreducible over $\mathbb{Z}_5$ and $\mathbb{Z}_5[x]/\langle x^2 + 2 \rangle$ is a field. \\

\textbf{[iii]} \underline{For $a = 3, f(x) = x^2 + 3$:}

$f(0) = (0 + 3) = 3 \not\in [0] \\
f(1) = (1 + 3) = 4 \not\in [0] \\
f(2) = (4 + 3) = 7 \not\in [0] \\
f(3) = (9 + 3) = 12 \not\in [0] \\
f(4) = (16 + 3) = 19 \not\in [0]$

Now, since $f(x)$ has no roots in $\mathbb{Z}_5$, it is irreducible over $\mathbb{Z}_5$ and $\mathbb{Z}_5[x]/\langle x^2 + 3 \rangle$ is a field. \\

\textbf{[iv]} \underline{For $a = 4, f(x) = x^2 + 4$:}

$f(0) = (0 + 4) = 4 \not\in [0] \\
f(1) = (1 + 4) = 5 \in [0]$

Now, since 1 is a root of $f(x)$, $f(x)$ is reducible over $\mathbb{Z}_5$ and $\mathbb{Z}_5[x]/\langle x^2 + 4 \rangle$ is not a field.