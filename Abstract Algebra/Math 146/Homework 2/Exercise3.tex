\subsection*{Exercise 14}
We want to show that the polynomial $f(x) = x^2 - 3$ has a root in $\mathbb{Q}(\sqrt{3})$ but not in $\mathbb{Q}(\sqrt{2})$ so that the two fields are not isomorphic. It is clear that $f(x)$ has a root in $\mathbb{Q}(\sqrt{3}) = \{a + b\sqrt{3} \mid a,b \in \mathbb{Q} \textnormal{ and } f(\sqrt{3}) = 0 \}$ since $f(\sqrt{3}) = (\sqrt{3})^2 - 3 = 0$.

Suppose for the sake of contradiction that $f(x)$ has a root in $\mathbb{Q}(\sqrt{2})= \{a + b\sqrt{2} \mid a,b \in \mathbb{Q} \textnormal{ and } f(\sqrt{2}) = 0 \}$. Then, $f(\sqrt{2}) = (\sqrt{2})^2 - 3 = 0$, which is absurd. Therefore, $f(x)$ cannot have a root in $\mathbb{Q}(\sqrt{2})$.

We claim that this implies that $\mathbb{Q}(\sqrt{3}) \not\cong \mathbb{Q}(\sqrt{2})$. Suppose for the sake of contradiction that $\mathbb{Q}(\sqrt{3}) \cong \mathbb{Q}(\sqrt{2})$. Then, there exists an isomorphism $\phi: \mathbb{Q}(\sqrt{3}) \rightarrow \mathbb{Q}(\sqrt{2})$ such that $\phi(\sqrt{3}) = \sqrt{3} = (a + b\sqrt{2}) \in \mathbb{Q}(\sqrt{2})$, for some $a,b \in \mathbb{Q}$. This implies that $b = 0$ and $a = \sqrt{3}$, which contradicts the fact that $a \in \mathbb{Q}$. Therefore, no such isomorphism exists between the two fields, and $\mathbb{Q}(\sqrt{3}) \not\cong \mathbb{Q}(\sqrt{2})$.