\subsection*{Exercise 13}
\textit{Use Exercise 11 to show that there are at most four distinct automorphisms of the field $\mathbb{Q}(\sqrt{2},\sqrt{3})$.}

\vspace{5 mm}
Any automorphism $\phi$ of $\mathbb{Q}(\sqrt{2},\sqrt{3})$ must be determined by the roots $\pm\sqrt{2}$ and $\pm\sqrt{3}$ by Exercise 6.4.11. We require that $\phi(a) = a$, for all $a \in \mathbb{Q}$ and $f(\phi(u)) = 0$ for any root $u \in \mathbb{Q}(\sqrt{2},\sqrt{3})$ of the given polynomial $f(x) \in \mathbb{Q}[x]$.

$\pm\sqrt{2}$ has minimal polynomial $x^2 - 2$ over $\mathbb{Q}$, and so $f(\pm\sqrt{2}) = 0 = f(\phi(\pm\sqrt{2}))$. Similarly, 
$\pm\sqrt{3}$ has minimal polynomial $x^2 - 3$ over $\mathbb{Q}$, and so $f(\pm\sqrt{3}) = 0 = f(\phi(\pm\sqrt{3}))$. This implies that $\phi(\sqrt{2}) = \pm\sqrt{2}$ and $\phi(\sqrt{3}) = \pm\sqrt{3}$, showing that there are at most four distinct automorphisms of $\mathbb{Q}(\sqrt{2},\sqrt{3})$, as required.