\subsection*{Exercise 1(d)}
\textit{Show that the following complex numbers are algebraic over $\mathbb{Q}$.}

\textit{(d) $\sqrt{2 + \sqrt{3}}$}

\vspace{5 mm}
Let $u = \sqrt{2 + \sqrt{3}} \in F = \mathbb{Q}(u)$, an extension field of $\mathbb{Q}$. We must find a nonzero polynomial $f(x) \in \mathbb{Q}[x]$ such that $f(u) = 0$. Since $u^2 = 2 + \sqrt{3}$ and $\sqrt{3}^2 = 3$, we can consider the polynomial $f(x) = (x^2 - 2)^2 - 3$. We see that if $f(x) = 0$, then $(x^2 - 2)^2 - 3 = 0 \implies (x^2 - 2)^2 = 3 \implies (x^2 - 2) = \sqrt{3} \implies x^2 = 2 + \sqrt{3} \implies x = \sqrt{2 + \sqrt{3}} = u$. Since $u$ satisfies the polynomial $f(x)= (x^2 - 2)^2 - 3$ in $\mathbb{Q}[x]$, $u$ is algebraic over $\mathbb{Q}$.