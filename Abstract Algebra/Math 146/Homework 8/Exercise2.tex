\subsection*{Exercise 2}
\textit{In Example 8.3.3 determine which subfields are conjugate, and in each case find an automorphism under which the subfields are conjugate.}

In $D_4$, we have that conjugation by $\alpha$ takes $\{1,\beta\}$ to the set $\{\alpha 1 \alpha^{-1}, \alpha\beta\alpha^{-1}\} = \{1,\alpha^2\beta\}$. Moreover, $\mathbb{Q}(\sqrt[4]{2})$ and $\mathbb{Q}(i\sqrt[4]{2})$ are conjugate, with respect to the automorphism given by $\alpha$ in Example 8.3.3.

On the other hand, in $D_4$, $\alpha$ takes the set $\{1,\alpha\beta\}$ to the set $\{1, \alpha^2\beta\alpha^{-1}\} = \{1,\alpha^3\beta\}$. Now, we have that $\mathbb{Q}(\sqrt[4]{2} - i\sqrt[4]{2})$ and $\mathbb{Q}(\sqrt[4]{2} + i\sqrt[4]{2})$ are conjugate with respect to the automorphism $\alpha$, as well.