\subsection*{Exercise 29}
Let $R$ be a commutative ring. We claim that the set of nilpotent elements form an ideal - called the nilradical of $R$, denoted by $\mathfrak{N}(R) = \{x \in R \mid x^m = 0$, for some $m \in \mathbb{Z}^+\}$. Note that since $R$ is commutative, the left ideals and right ideals coincide. Then, we must verify that $\mathfrak{N}(R)$ is a subring of $R$ such that $r\mathfrak{N}(R) = \mathfrak{N}(R)r \subseteq \mathfrak{N}(R)$, for every $r \in R$.

$\mathfrak{N}(R) \subseteq R$ is a subring of $R$ if it is a subgroup of $R$ that is closed under multiplication. It suffices for us to check that $\mathfrak{N}(R) \not= \emptyset$ and is closed under multiplication and division. Since $0 \in R$ satisfies $0^m = 0$, for any $m \in \mathbb{Z}^+$, it follows that $0 \in \mathfrak{N}(R) \not= \emptyset$. Note that if $\mathfrak{N}(R) \not= \{0\}$, then it contains at least one $x \in R$ which is a zero divisor by Exercise 7.1.14. Since the product of zero divisors is zero, and since the product of zero divisors and zero is zero, it follows that $\mathfrak{N}(R)$ is closed under multiplication. To show that $\mathfrak{N}(R)$ is closed under addition, consider some $a,b \in \mathfrak{N}(R)$. Then, let $a^m = 0 = b^n$, for some $m,n \in \mathbb{Z}$. Then, since $R$ is commutative and by Exercise 7.3.26 and Proposition 7.1.1, $(a + b)^{(m+n)} = (a^{(m+n)} + b^{(m+n)}) = (a^ma^n + b^mb^n) = (0a^n + b^m0) = (0 + 0) = 0$. That is, the sum of nilpotent elements ($a + b)$ is nilpotent in $R$ as well, so $\mathfrak{N}(R)$ is a subring of $R$.

Now, we verify that $r\mathfrak{N}(R) = \mathfrak{N}(R)r \subseteq \mathfrak{N}(R)$, for each $r \in R$. But by Exercise 7.1.14, if $x \in \mathfrak{N}(R)$, then $rx = xr \in \mathfrak{N}(R)$, for each $r \in R$. Thus, $r\mathfrak{N}(R) = \mathfrak{N}(R)r = \mathfrak{N}(R) \subseteq \mathfrak{N}(R)$, so $\mathfrak{N}(R)$ is an ideal of $R$.