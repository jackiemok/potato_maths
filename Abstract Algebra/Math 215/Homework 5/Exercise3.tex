\subsection*{Exercise 26}
The characteristic of a ring $R$ is the smallest $n \in \mathbb{Z}^+$ such that $n1 = 0 \in R$. If no such $n$ exists, then the characteristic of $R$ is said to be 0. For example, $\mathbb{Z}/n\mathbb{Z}$ is a ring of characteristic $n$, for each $n \in \mathbb{Z}^+$, and $\mathbb{Z}$ is a ring of characteristic 0.

\vspace{3 mm}

\textbf{[a]} Consider the map $\phi: \mathbb{Z} \rightarrow R$ defined as follows:
\[
\phi(k) =
\begin{cases}
    \hspace{1.5 mm} 1 + 1 + ... + 1 \hspace{2 mm} ( k \textnormal{ times} ) & \textnormal{ if } k > 0 \\
    \hspace{1.5 mm} 0 & \textnormal{ if } k = 0 \\
    \hspace{1.5 mm} -1 - 1 - ... - 1 \hspace{2 mm} ( k \textnormal{ times} ) & \textnormal{ if } k < 0
\end{cases}
\]
We claim that $\phi$ is a ring homomorphism with ker($\phi) = n\mathbb{Z}$, where $n$ is the characteristic of $R$. Now, we consider for $a,b \in \mathbb{Z}$, the values $\phi(ab)$ and $\phi(a+b)$ to show that $\phi$ is both additive and multiplicative. By Proposition 7.1.1, we obtain the following:
\[
\phi(a+b) =
\begin{cases}
    \phi(0 + 0) = \phi(0) = 0 = 0 + 0 = \phi(0) + \phi(0) & \textnormal{if } a,b = 0 \\
    \phi(0 + b) = \phi(b) = b1 = b = 0 + b = \phi(0) + \phi(b) & \textnormal{if } a = 0, b > 0 \\
    \phi(0 + b) = \phi(b) = b(-1) = -b = 0 + -b = \phi(0) + \phi(b) & \textnormal{if } a = 0, b < 0 \\
    \\
    \phi(a + b) = (a + b)1 = a1 + b1 = a + b = \phi(a) + \phi(b) & \textnormal{if } a,b > 0 \\
    \phi(a + b) = (a + b)1 = a1 - b1 = a1 + b(-1) = a + -b = \phi(a) + \phi(b) & \textnormal{if } a > 0, b < 0 \\
    \phi(a + b) = (a + b)(-1) = a(-1) + b(-1) = -a + -b = \phi(a) + \phi(b) & \textnormal{if } a,b  < 0
\end{cases}
\]
This shows that $\phi$ is indeed additive.
\[
\phi(ab) =
\begin{cases}
    \phi(0\cdot0) = \phi(0) = 0 = 0\cdot0 = \phi(0)\phi(0) & \textnormal{if } a,b = 0 \\
    \phi(0\cdot b) = \phi(0) = 0 = 0\cdot b = \phi(0)\phi(b) & \textnormal{if } a = 0, b > 0 \\
    \phi(0\cdot b) = \phi(0) = 0 = 0\cdot-b = \phi(0)\phi(b) & \textnormal{if } a = 0, b < 0 \\
    \\
    \phi(ab) = (ab)1 = a1\cdot b1 = a\cdot b = \phi(a)\phi(b) & \textnormal{if } a,b > 0 \\
    \phi(ab) = (ab)(-1) = a\cdot b(-1) = a \cdot -b = \phi(a)\phi(b) & \textnormal{if } a > 0, b < 0 \\
    \phi(ab) = (ab)1 = a1\cdot b1 = a(-1) \cdot b(-1) = -a\cdot -b = \phi(a)\phi(b) & \textnormal{if } a,b < 0
\end{cases}
\]
This shows that $\phi$ is indeed multiplicative, and $\phi$ is indeed a homomorphism. 

Now, consulting the above results and recalling our assumption that $R$ has characteristic $n \in \mathbb{Z}^+$, we obtain that ker$(\phi) = \{ a \in \mathbb{Z} \mid \phi(a) = 0 \in R\} = \{0\} \cup \{nq \mid q \in \mathbb{Z}^+\} \cup \{nk \mid k \in \mathbb{Z}^-\}$ since $\phi(0) = 0$, $\phi(nq) = \phi(n)\phi(q) = n1\phi(q) = 0\phi(q) = 0$, and $\phi(nk) = \phi(n)\phi(k) = n1\phi(k) = 0\phi(k) = 0$, for $q \in \mathbb{Z}^+$ and $k \in \mathbb{Z}^-$. Now, since $\phi(nm) = 0$, for any $m \in \mathbb{Z}$, ker$(\phi) = n\mathbb{Z}$, where $n$ is the characteristic of $R$.

\vspace{3 mm}

\textbf{[b]} We now determine the characteristics of the rings $\mathbb{Q}, \mathbb{Z}[x],$ and $ \mathbb{Z}/n\mathbb{Z}[x]$. $\mathbb{Q}$ is a field and thus an integral domain, so it has no zero divisors. Therefore, $\mathbb{Q}$ has characteristic 0. Similarly, $\mathbb{Z}$ is an integral domain, and therefore by Proposition 7.2.4, $\mathbb{Z}[x]$ is an integral domain which has no zero divisors. This again implies that $\mathbb{Z}[x]$ has characteristic 0.

Now, consider $\mathbb{Z}/n\mathbb{Z}[x]$. Since $\mathbb{Z}_n \cong \mathbb{Z}/n\mathbb{Z}$ by the natural projection $\pi: \mathbb{Z} \rightarrow \mathbb{Z}_n$ defined by $\pi(x) = \overline{x}$, for all $x \in \mathbb{Z}$, we have that ker($\pi) = n\mathbb{Z}$. Now, by part (a), we have that $\mathbb{Z}/n\mathbb{Z}$ has characteristic $n$. This implies that $\mathbb{Z}/n\mathbb{Z}[x]$ has characteristic $n$ as well, since $f(x) \equiv 0 \in \mathbb{Z}[x]$ when its coefficients are all identically 0 or multiples of $n$ in $\mathbb{Z}/n\mathbb{Z}$.

\vspace{3 mm}

\textbf{[c]} Suppose that $R$ is a commutative ring of characteristic $p$, for some prime number $p$. We claim that $(a + b)^p = (a^p + b^p)$, for all $a,b \in R$. Since $R$ is commutative and has characteristic $p$, we deduce that $p1 = 0 \implies a(p1) = (p1)a = 0a = 0$, for any $a \in R$ by Proposition 7.1.1. By the Binomial Theorem:
\begin{align*}
    (a + b)^p &= \sum_{i=0}^p {p \choose i} a^ib^{p-i} \\
    &= b^p + pab^{p-1} + {p \choose 2}a^2b^{p-2} + ... + {p \choose p-2}a^{p-2}b^2 + pa^{p-1}b + a^p \\
    &= b^p + pab^{p-1} + \frac{p!}{2!(p-2)!}a^2b^{p-2} + ... + \frac{p!}{2!(p-2)!}a^{p-2}b^2 + pa^{p-1}b + a^p \\
    &= b^p + p(ab^{p-1} + \frac{(p-1)!}{2!(p-2)!}a^2b^{p-2} + ... + \frac{(p-1)!}{2!(p-2)!}a^{p-2}b^2 + a^{p-1}b) + a^p \\
    &= b^p + p\sum_{i=1}^{p-1}\frac{(p-1)!}{i!(p-i)!}a^ib^{p-i} + a^p \\
    &= b^p + 0 + a^p = (a^p + b^p).
\end{align*}
Therefore, $(a + b)^p = (a^p + b^p)$, for all $a,b \in R$.