\subsection*{Exercise 4}
We want to show that every finite abelian group is a torsion $\mathbb{Z}$-module. Suppose that $G$ is a finite abelian group of order $n \in \mathbb{Z}^+$. Then, $G$ is also a $\mathbb{Z}$-module of order $n$. By Lagrange's Theorem, every element of $G$ has finite order dividing $n$, so $na = 0$, for any $a \in G$. Since $n \in \mathbb{Z}^+ \subset \mathbb{Z}$, $na \in G$ can also be viewed as the action of $n$ on $a \in G$, where we know $G$ to be a $\mathbb{Z}$-module. Since $na = 0$ for each $a \in G$, $a \in$ Tor($G)$ for every $a \in G$, which implies that $G \subseteq$ Tor($G)$. But, by the definition of the torsion elements of $G$, we also have that Tor($G) \subseteq G$, so that indeed Tor($G) = G$.

An example of an infinite abelian group that is a torsion $M = \mathbb{Z}$-module is $\prod_{i \in \mathbb{Z}^+} \mathbb{Z}_p = (\mathbb{Z}_p \times \mathbb{Z}_p \times ...)$, for some prime $p$. By Lagrange's Theorem, every element of $\mathbb{Z}_p$ must divide $p$, so $p\overline{x} = \overline{0}$, for every element $\overline{x}$. Since by Exercise 7.3.26 of Homework 5, $\mathbb{Z}_n$ has characteristic $n$, for each $n \in \mathbb{Z}^+$, and in view of Exercise 10.1.11, we see that the elements $p\mathbb{Z}$ annihilate $M$. Combining these facts, we obtain that Tor$(M) = M$.