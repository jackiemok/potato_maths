\subsection*{Exercise 5}
Let $R$ be an integral domain. We claim that every finitely generated torsion $R$-module has a nonzero annihilator. Let $M$ be a finitely generated torsion $R$-module. Then, Tor($M) = M$ and for some subset $A = \{a_1,...,a_m\} \subseteq M$, we have $M = RA = \{\sum_{i=1}^m r_ia_i \mid r_1,...,r_m \in R, a_1,...,a_m \in A\}$. Since $M$ is a torsion $R$-module, we can also find nonzero $q_i \in R$, such that $q_ia_i = 0$, for each $i = 1,...,m$. Since $R$ is an integral domain, $\prod_{i=1}^m q_i =: q \not= 0$ and $qra_i = rqa_i = r\prod_{j\not=i}q_j(q_ia_i) = r\prod_{j\not=i}q_j0 = 0$, for each $i = 1,...,m$ and $r \in R$. Now, consider some element $a \in M$, where $a = \sum_{i=1}^mr_ia_i$, for some $r_i \in R, a_i \in A, i=1,...,m$. Then, $qa = q\sum_{i=1}^mr_ia_i = q(r_1a_1 + ... + r_ma_m) = qr_1a_1 + ... + qr_ma_m = (qr_1)a_1 + ... + (qr_m)a_m = 0$, from our previous result. Then, $q \not=0$ is our desired annihilator.

An example of a torsion $R=\mathbb{Z}$-module whose annihilator is the zero ideal is given by $M = \mathbb{Q}/\mathbb{Z}$, since we cannot find a nonzero $x \in \mathbb{Z}$ that has all possible denominators as factors. Therefore, only the zero ideal annihilates $M$.