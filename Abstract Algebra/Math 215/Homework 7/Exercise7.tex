\subsection*{Exercise 3}
We must show that the $F[x]$ modules in Exercises 10.1.18 and 10.1.19 are both cyclic. In both cases, $F = \mathbb{R}$ and $V$ is an $F[x]$ module via the linear transformations $T$. $V$ is cyclic if and only if every element of $V$ can be written as an $F$-linear combination of elements of the set $\{T^n(v) \mid n \ge 0\}$.

\vspace{5 mm}
[1] \textit{Exercise 10.1.18}

$V = \mathbb{R}^2$ and $T: \mathbb{R}^2 \rightarrow \mathbb{R}^2$ is clockwise rotation about the origin by $\pi/2$ radians. Then, $T^n$ is clockwise rotation about the origin by $n\pi/2$ radians, for $n \in \mathbb{Z}$. Therefore, for $n \in 4\mathbb{Z}$, $T^n$ is the identity, so $T^n(v) = v$, for each $v \in V$. Since any element in $V$ can be expressed as $v = (x,y)$, for some $x,y \in \mathbb{R}$, we can describe $T$ by $T(v) = T(x,y) = (y,-x),$ where $(y,-x) = y(1,0) - x(0,1)$.

\vspace{5 mm}
[2] \textit{Exercise 10.1.19}

$V = \mathbb{R}^2$ and $T: \mathbb{R}^2 \rightarrow \mathbb{R}^2$ is projection onto the $y$-axis. Since any element in $V$ can be expressed as $v = (x,y)$, for some $x,y \in \mathbb{R}$, we can describe $T$ by $T(v) = T(x,y) = (0,y)$, where $(0,y) = y(0,1)$.
