\subsection*{Exercise 2(c)}
We want the list of invariant factors for all abelian groups of order $320 = 2^6\cdot5$. Let $G$ be an abelian group of order 320. By Theorem 5, $G \cong A_1 \times A_2$, where $|A_1| = 2^6 = 64$ and $|A_2| = 5^1 = 5$.

We obtain the following list of abelian groups from the list of positive integer partitions of 6:
\begin{align}
\setcounter{equation}{0}
    6 &= 1+1+1+1+1+1 &&\implies A_1 \cong \mathbb{Z}_2 \times \mathbb{Z}_2 \times \mathbb{Z}_2 \times \mathbb{Z}_2 \times \mathbb{Z}_2 \times \mathbb{Z}_2 \\
    &= 2+1+1+1+1 &&\implies A_1 \cong \mathbb{Z}_4 \times \mathbb{Z}_2 \times \mathbb{Z}_2 \times \mathbb{Z}_2 \times \mathbb{Z}_2 \\
    &= 2+2+1+1 &&\implies A_1 \cong \mathbb{Z}_4 \times \mathbb{Z}_4 \times \mathbb{Z}_2 \times \mathbb{Z}_2 \\
    &= 2+2+2 &&\implies A_1 \cong \mathbb{Z}_4 \times \mathbb{Z}_4 \times \mathbb{Z}_4 \\
    &= 2+3+1 &&\implies A_1 \cong \mathbb{Z}_4 \times \mathbb{Z}_8 \times \mathbb{Z}_2 \\
    &= 2+4 &&\implies A_1 \cong \mathbb{Z}_4 \times \mathbb{Z}_{16} \\
    &= 3+1+1+1 &&\implies A_1 \cong \mathbb{Z}_8 \times \mathbb{Z}_2 \times \mathbb{Z}_2 \times \mathbb{Z}_2 \\
    &= 3+3 &&\implies A_1 \cong \mathbb{Z}_8 \times \mathbb{Z}_8 \\
    &= 4+1+1 &&\implies A_1 \cong \mathbb{Z}_{16} \times \mathbb{Z}_2 \times \mathbb{Z}_2 \\
    &= 5+1 &&\implies A_1 \cong \mathbb{Z}_{32} \times \mathbb{Z}_2 \\
    &= 6 &&\implies A_1 \cong \mathbb{Z}_{64}
\end{align}
On the other hand, since 1 is the exponent for $5$ in the prime decomposition of $|G| = 320$, we only have the case $A_2 \cong \mathbb{Z}_5$ to consider. Therefore, we have $(11 \cdot 1) = 11$ abelian groups of order 320, up to isomorphism, and thus, 11 invariant factors given below. Note that the invariant factors have been ordered so that in a given sequence of the form $n_1, n_2,..., n_k$, we have $n_{i+1} | n_{i}$. 
\begin{align*}
 &\textbf{Elementary Divisors} && \textbf{Invariant Factors} \\
 &\textbf{2}, 2, 2, 2, 2, 2, \textbf{5} && \textbf{10}, 2, 2, 2, 2, 2 \\
 &\textbf{4}, 2, 2, 2, 2, \textbf{5} && \textbf{20}, 2, 2, 2, 2 \\
 &\textbf{4}, 4, 2, 2, \textbf{5} && \textbf{20}, 4, 2, 2 \\
 &\textbf{4}, 4, 4, \textbf{5} && \textbf{20}, 4, 4 \\
 &4, \textbf{8}, 2, \textbf{5} && \textbf{40}, 4, 2 \\
 &4, \textbf{16}, \textbf{5} && \textbf{80}, 4 \\
 &\textbf{8}, 2, 2, 2, \textbf{5} && \textbf{40}, 2, 2, 2 \\
 &\textbf{8}, 8, \textbf{5} && \textbf{40}, 8 \\
 &\textbf{16}, 2, 2, \textbf{5} && \textbf{80}, 2, 2 \\
 &\textbf{32}, 2, \textbf{5} && \textbf{160}, 2 \\
 &\textbf{64}, \textbf{5} && \textbf{320}
\end{align*}
