\subsection*{Exercise 4}
We want the commutator subgroups of $S_4$ and $A_4$, where $A_4 \triangleleft S_4$. $S_4/A_4$ is abelian since $|S_4/A_4| = 2$ is prime. Then, by Proposition 7, $S_4' \le A_4$ since $S_4/S_4'$ is the largest abelian quotient of $S_4$. Note that $S_4' \not=1$ since $S_4$ is not abelian. By Exercise 4.4.8 and Proposition 7, since $S_4'$ char $S_4$ and thus, $S_4'$ char $A_4 \triangleleft S_4$, it follows that $S_4' \triangleleft S_4$. Therefore, $S_4' \trianglelefteq A_4$ is the smallest nontrivial normal subgroup of $A_4$ such that the corresponding quotient group is abelian. Now, by Exercise 4.4.8, the Klein 4-group $V_4$, which is characteristic in $S_4$ and thus $A_4$, is the minimal normal subgroup of $A_4$ since the cyclic subgroups isomorphic to $\mathbb{Z}_2$ are not normal in $A_4$. However, $S_4/V_4$ is not abelian since $|S_4/V_4| = 4!/4 = 3! = 6$ and $S_4/V_4 \cong S_3$. Moreover, there are four subgroups of order 3, so these cannot be characteristic in $S_4$. Therefore, we are left with $A_4 \triangleleft S_4$, where $S_4/A_4$ is abelian, and so $S_4' = A_4$.

It now remains for us to find $A_4'$. Again, by Proposition 7, $A_4'$ char $A_4$ and $A_4/A_4'$ is the largest abelian quotient of $A_4$. Therefore, $A_4'$ is the smallest characteristic subgroup of $A_4$ such that the corresponding quotient group is abelian. Note that $A_4' \not= 1$ since $A_4$ is not abelian. Since there are three subgroups of order 2, they are not characteristic. By Exercise 4.4.8, the Klein 4-group $V_4$, which is characteristic in $S_4$ and thus $A_4$, is the minimal normal subgroup of $A_4$ since the cyclic subgroups isomorphic to $\mathbb{Z}_2$ are not normal in $A_4$. Then, $A_4/V_4$ is abelian since $|A_4/V_4| = (4!/2)/4 = 12/4 = 3$ is prime, so $A_4' = V_4$.