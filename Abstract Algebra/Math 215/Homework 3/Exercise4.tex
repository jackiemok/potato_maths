\subsection*{Exercise 9}
Let $D_{2n} = \langle r,s \mid r^n = e = s^2, rs = sr^{-1} \rangle$. We use the preceding exercises to show that any subgroup of $\langle r \rangle$ is normal in $D_{2n}$. First, note that $\langle r \rangle \trianglelefteq D_{2n}$ since $|D_{2n}/\langle r \rangle| = 2n/n = 2$. Moreover, since $\langle r \rangle$ is cyclic, any subgroup $H$ is characteristic in $\langle r \rangle$. That is, any subgroup $H$ in $\langle r \rangle$ is of the form $\langle r^i \rangle$, where $i \in \mathbb{Z}$, and $\langle r^i \rangle = \langle r \rangle$ whenever $(i,n) = 1$. Similarly, if $|\langle r^i \rangle| = |\langle r^j \rangle|$, where $i \not= j$, then the exponents of the elements in each of the subgroups are equivalent modulo $n$, so that they both define the same subgroups in $\langle r \rangle$. Therefore, subgroups in $\langle r \rangle$ are uniquely determined by their orders dividing $n$, the order of $\langle r \rangle$. Any $\phi \in$ Aut($\langle r \rangle$) must therefore send $H$ to itself to preserve order. Thus, $H \le \langle r \rangle \implies H$ char $\langle r \rangle \trianglelefteq D_{2n}$. It now follows from the previous exercise that if $H \le \langle r \rangle$, then $H \trianglelefteq D_{2n}$.