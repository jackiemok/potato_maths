\subsection*{Exercise 19}
Let $H$ be some group with $h \in H$. We want to show that there is a unique homomorphism $\phi: \mathbb{Z} \rightarrow H$ such that $\phi(1) = h$. We must choose $\phi$ so that $\phi(0) = e_H$ and so that $\forall m,n \in \mathbb{Z}, \phi(m + n) = \phi(m)\phi(n)$. Moreover, since $\mathbb{Z} = \langle1\rangle$, choose $\phi$ so that it maps surjectively to the subgroup $\langle h \rangle \le H$. Then, for any $n \in \mathbb{Z}, \phi(n) = \phi(n\cdot 1) = \phi(1)^n = h^n \in \langle h \rangle$. That is, if $|h| = N < \infty$, then for $i \in \mathbb{Z}, \phi((N+i)\cdot1) = \phi(1)^{(N+i)} = h^{(N+i)} = h^Nh^i = e_H\cdot h^i = h^i$, which is $e_H \iff i \equiv 0$ (mod $N$). Otherwise, if $|h| = \infty$, then $\phi$ is still multiplicative. We have shown that there exists such a homomorphism $\phi$, and now it remains for us to verify that $\phi$ is unique. But since 1 is the generator for $\mathbb{Z}$, this result automatically follows.