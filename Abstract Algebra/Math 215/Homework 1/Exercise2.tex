\subsection*{Exercise 7}
Let $a,b \in \mathbb{Z}$, and consider the sum of their squares $(a^2 + b^2)$. We claim that $(a^2 + b^2)$ never leaves a remainder of 3 when divided by 4. To see this, first write $(a^2 + b^2) = 4q + r$, for some $q, r \in \mathbb{Z}$ with $0 \le r < 4$. That is, $(a^2 + b^2) - r = 4q$, or equivalently, $(a^2 + b^2) \equiv r$ (mod 4). Then, $[a^2 + b^2] = [a]^2 + [b]^2 = [r] \in \mathbb{Z}/4\mathbb{Z}$. But by the previous exercise, $[a]^2, [b]^2 \in \{[0],[1]\}$ necessarily, which implies that $r \in \{[0],[1],[2]\}$, our desired result. 