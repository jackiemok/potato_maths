\subsection*{5.}
Using our result from the previous exercise, we now classify all groups of order $75 = 3\cdot5^2$. Let $G$ be a group of order 75. We first consider the case when $G$ is abelian. By the Fundamental Theorem of Finitely Generated Abelian Groups, $G \cong A_1 \times A_2$, where $|A_1| = 3$ and $|A_2| = 5^2 = 25$. Since 1 is the exponent of 3 in the prime decomposition of $|G|=75$, we have only the case that $A_1 \cong \mathbb{Z}_3$. On the other hand, since 2 is the exponent of 5 in the prime decomposition of 75, we have the two cases: $A_2 \cong \mathbb{Z}_{25}$ and $A_2 \cong \mathbb{Z}_5\times\mathbb{Z}_5$. Combining these results and by Proposition 5.2.6, we have that $G \cong \mathbb{Z}_3\times\mathbb{Z}_{25} \cong \mathbb{Z}_{75}$, since $(3,25)=1$, and $G \cong \mathbb{Z}_3 \times \mathbb{Z}_5 \times \mathbb{Z}_5 \cong \mathbb{Z}_{15}\times\mathbb{Z}_5$, since $(3,5)=1$. Thus, the abelian groups of order 75, up to isomorphism, are $\mathbb{Z}_{75}$ and $\mathbb{Z}_{15}\times\mathbb{Z}_5$.


Now, suppose that $G$ is not abelian. The number of Sylow 5-subgroups of $G$ is given by $n_5 \equiv 1$ (mod 5), where $n_5 \mid 3$, so that $n_5 = 1$. Thus, for $P \in Syl_5(G)$, $P \triangleleft G$. Similarly, the number of Sylow 3-subgroups of $G$ is given by $n_3 \equiv 1$ (mod 3), where $n_3 \mid 5^2$, so that $n_3 = 1$ or 25. Let $Q \in Syl_3(G)$. Then, $P \cap Q = \{e\}$ by Lagrange's Theorem, which implies that $G = PQ = P \rtimes_\phi Q$, for some $\phi:Q \rightarrow$ Aut($P$), by Theorem 5.5.12. Now, by our above computation for $A_2$, we deduce that there are two subgroups of order 25, so that $P \cong \mathbb{Z}_{25}$ or $P \cong \mathbb{Z}_5 \times \mathbb{Z}_5$.

If $P \cong \mathbb{Z}_{25}$, then Aut($P) \cong$ Aut($\mathbb{Z}_{25}) \cong \mathbb{Z}_{25}^\times$, which has order $\phi(25) = 25(1-\frac{1}{5}) = 20$ by Proposition 4.4.16. But by Lagrange's Theorem, there cannot exist any elements in Aut($P)$ of order 3, thereby forcing $\phi$ to be the trivial map. This implies that $P \rtimes_\phi Q \cong \mathbb{Z}_{25}\times\mathbb{Z}_3 \cong\mathbb{Z}_{75}$, which is abelian, contradicting our assumption that $G$ is not abelian. Then, we may assume that $P \cong \mathbb{Z}_5 \times \mathbb{Z}_5$.

$P \cong \mathbb{Z}_5 \times \mathbb{Z}_5$ implies that Aut($P) \cong$ Aut($\mathbb{Z}_5 \times \mathbb{Z}_5) \cong GL_2(\mathbb{F}_5)$ by Proposition 4.4.17. Note that $|GL_2(\mathbb{F}_5)| = (5^2-1)(5^2-5) = 2^5\cdot3\cdot5 = 480$ by the results given on page 35. Moreover, the Sylow 3-subgroups of Aut($P$) have order 3, which are cyclic and abelian, and all are conjugate to one another, so $\phi$ must be nontrivial with ker$(\phi) = e$. Now, since $\phi$ is injective whose image is a Sylow 3-subgroup of Aut($P$), $(\mathbb{Z}_{5}\times\mathbb{Z}_5) \rtimes_\phi \mathbb{Z}_3$ gives us nonabelian $G$ of order 75.