\newpage
\subsection*{1.}

\textbf{(a)}
Let $p$ be a prime number and consider the symmetric group on $p$ objects, $S_p$. Suppose that $P \le S_p$ with $|P| = p$. We claim that $|N_{S_p}(P)| = p(p-1)$. 

Let $P = \langle \tau \rangle$, for some $\tau \in S_p$ of order $p$. By the Cycle Decomposition Algorithm for Symmetric Groups and since the order of a permutation is the least common multiple of the lengths of the cycles in its cycle decomposition, we may write $\tau = (\tau_1 \hspace{1mm} \tau_2 \hspace{1mm} ... \hspace{1mm} \tau_p)$ and $P = \langle (\tau_1 \hspace{1mm} \tau_2 \hspace{1mm} ... \hspace{1mm} \tau_p) \rangle$. By Lagrange's Theorem, any $\tau^i \in P$, for $i \in \mathbb{Z}$, either has order 1, in which case $\tau^i = (1)$, or $p$, in which case $\tau^i$ is a $p$-cycle as well.

By Proposition 4.3.6, the number of conjugates of $P$ in $S_p$ is given by the index of the normalizer of $P$, $|S_p:N_{S_p}(P)|$. Furthermore by Proposition 4.3.11, $\sigma \in S_p$ is conjugate to $\tau^i \in P$, for some $i \in \mathbb{Z}$, if and only if $\sigma$ and $\tau^i$ have the same cycle type. Then, the elements of $S_p$ which are conjugate to elements of $P$ are $p$-cycles. There are $p!$ ways to arrange $p$ objects into a $p$-cycle, each of which has $p$ equivalent representations, ie., $(a_1 \hspace{1mm} a_2 \hspace{1mm} ... \hspace{1mm} a_p) = (a_2 \hspace{1mm} a_3 \hspace{1mm} ... \hspace{1mm} a_p \hspace{1mm} a_1) = ... = (a_p \hspace{1mm} a_1 \hspace{1mm} ... \hspace{1mm} a_{p-1})$. Thus, there are $p!/p = (p-1)!$ distinct $p$-cycles in $S_p$, each of which generates a subgroup of order $p$. Then, for each of these $p$-subgroups $P_i$, $i = 1,...,(p-1)!$, $P_i$ is generated by any of the $(p-1)$ elements $\sigma \in P_i \setminus \{(1)\}$. Now, we have $(p-1)!/(p-1) = (p-2)!$ cyclic subgroups of order $p$, which are conjugate to $P$. This implies that $|S_p:N_{S_p}(P)| = (p-2)! \implies |N_{S_p}(P)| = |S_p|/(p-2)! = p!/(p-2)! = p(p-1)$.
\vspace{5mm}

\textbf{(b)} 
By Corollary 4.4.15, $N_{S_p}(P)/C_{S_p}(P) \cong K \le$ Aut($P$), for some subgroup $K \le$ Aut($P$). We claim that $K =$ Aut($P$). By Proposition 4.4.17, if $p$ is an odd prime, then $|$Aut($P)| = (p-1)$. 

Furthermore, in part (a), we obtained that $|N_{S_p}(P)| = p(p-1)$. By the definitions of $C_{S_p}(P) = \{\sigma \in S_p \mid \sigma\tau\sigma^{-1} = \tau$, for all $\tau \in S_p\}$ and $N_{S_p}(P) = \{\sigma \in S_p \mid \sigma P \sigma^{-1} = P\}$, we have that $C_{S_p}(P) \le N_{S_p}(P)$. Thus, by Lagrange's Theorem, $|C_{S_p}(P)|$ must divide $|N_{S_p}(P)| = p(p-1)$, and $|N_{S_p}(P):C_{S_p}(P)|$ must divide $|$Aut($P)| = (p-1)$. By Exercise 2.2.6(b)$^*$, since $P = \langle \sigma \rangle$, for some $\sigma \in S_p$ of order $p$, is cyclic and thus abelian, $P \le C_{S_p}(P)$. Then, $|C_{S_p}(P)| = p$ or $p(p-1)$ since $C_{S_p}(P) \le N_{S_p}(P)$, but since $|N_{S_p}(P):C_{S_p}(P)| = p(p-1)/|C_{S_p}(P)|$ must divide $|$Aut($P)| = (p-1)$, it follows that $|C_{S_p}(P)| = p$. Now, $|N_{S_p}(P):C_{S_p}(P)| =$ $|$Aut($P)|$, so we conclude that $N_{S_p}(P)/C_{S_p}(P) \cong$ Aut($P$), as desired.

Note that if $p = 2$ is even, then $P = S_2 = \{(1),(12)\} = \langle (12) \rangle \cong \mathbb{Z}_2$, whose automorphism group has order given by Euler's totient function $\phi(2) = (2-1) = 1$ by Proposition 4.4.16. We have $N_{S_2}(P) = N_{S_2}(S_2) = \{\sigma \in S_2 \mid \sigma S_2 \sigma^{-1} = S_2\} = S_2$, and similarly, $C_{S_2}(P) = C_{S_2}(S_2) = S_2$ since $Z(S_2) \le C_{S_2}(S_2)$, where $S_2$ is abelian. Therefore, $|N_{S_2}(P):C_{S_2}(P)| = 2/2 = 1 = |$Aut$(P)|$, and $N_{S_2}(P)/C_{S_2}(P) \cong$ Aut$(P)$ as well.

\vspace{5mm}
\underline{\textbf{$^*$Lemma: Exercise 2.2.6(b)}}

\textit{Let $H \le G$. Then, $H \le C_G(H) \iff H$ is abelian.}

\textbf{Proof:}

$H \le C_G(H) = \{g\in G\mid ghg^{-1} = h$, for all $h \in H\} \implies h'h = hh'$, for all $h',h \in H$. Then, $H$ is abelian. Conversely, suppose that $H$ is abelian. Let $g \in H$. Then, $gh = hg$, for all $h \in H$, which implies that $g \in C_G(H)$. But since $g$ has been chosen arbitrarily, it follows that $H \le C_G(H)$.