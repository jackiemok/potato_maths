\subsection*{3.}
Let $G$ be a simple group of order $168 = 2^3 \cdot 3 \cdot 7$. By Sylow's Theorem, $Syl_7(G) \not= \emptyset$ since $7 \nmid m = (2^3 \cdot 3)$ and $7m = |G| = 168$. Furthermore, the number of Sylow 7-subgroups of $G$ is given by $n_7 \equiv 1$ (mod 7), where $n_7 \mid m$. Then, since $a \not\equiv 1$ (mod 7), for any $a \in \{3,(2^3\cdot3)=m=24\}$, we require that $n_7 = 1$ or $2^3=8$. But since $G$ is simple, $n_7 \not= 1$ since otherwise there would be a unique Sylow 7-subgroup normal in $G$. Therefore, the number of Sylow 7-subgroups in $G$ is $n_7 = 8$.

We want to know how many elements in $G$ have order 7, and by Lagrange's Theorem, these elements must belong to subgroups of $G$ in $Syl_7(P)$ or some other 7-subgroups of $G$. Note that given any $P_i \in Syl_7(P) = \{P_i \le G \mid |P_i| = 7,$ for $i = 1,...,8\}$, we have that $P_i = \langle a_i \rangle,$ for some $a_i \in G$ with order 7, since $P_i$ is cyclic of prime order. We require that $P_i \cap P_j = \{e\}$, for each $i \not= j \in \{1,...,8\}$, since each non-identity element of a Sylow 7-subgroup of $G$ has order 7 by the Euler totient function $\phi(7) = (7-1) = 6$. Thus, if any two Sylow 7-subgroups have nontrivial intersection, then they both define the same Sylow 7-subgroup. Therefore, there are 6 elements of order 7 in each of the 8 Sylow 7-subgroups, which implies that there are $(6\cdot8)=48$ elements of order 7 contained in the subgroups of $G$ in $Syl_7(G)$.

Moreover, for any $k \ge 2$, $7^k \nmid |G|=168$, so by Lagrange's Theorem, there cannot exist any subgroups in $G$ of order $7^kq$, for any $q \in \mathbb{Z}$. Thus, we have no additional elements in $G$ of order 7, and we conclude that $G$ contains 48 elements of order 7.