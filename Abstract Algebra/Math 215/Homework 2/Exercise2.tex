\subsection*{Exercise 6}
Let $G$ be a finite group with $G \not= \{e\}$. We proceed by induction on $|G|$ to show the existence of a composition series for $G$. In the case where $|G| = 1$, we have the trivial group $G = \{e\}$, so instead consider the case where $|G| = |\{e,a\}| = 2$. Then, it is certainly true that $\{e\} = N_0 \trianglelefteq N_1 = \{e,a\} = G$, where $G/\{e\} = G$ is simple. Now, suppose that the statement holds for some $n \in \mathbb{N}$. 

Then, suppose that for a group $G$ with $|G| = (n=1)$, we have $1 = N_0 \trianglelefteq N_1 \trianglelefteq ... \trianglelefteq N_m = G$, for some $m \ge 1$. Such an $m$ exists indeed, since we can always take $m = 1$, in which case $G$ would be simple. Otherwise, $m > 1$ and $G$ is not simple, and we can suppose that this sequence of subgroups, which are normal in the following subgroup, is the longest such chain of subgroups. Then, it remains for us to verify that $N_{i+1}/N_i$ is simple, for each $0 \le i \le (m-1)$. 

Suppose for the sake of contradiction that $N_j/N_{j-1}$ is not simple, for some $1 \le j \le m$. Then, there exists some $\overline{K} \trianglelefteq N_j/N_{j-1},$ where $\overline{K} \not = N_j/N_{j-1},\{e\}$. By the Third Isomorphism Theorem, this $\overline{K}$ corresponds to the subgroup $K$ such that $N_{j-1} \trianglelefteq K \trianglelefteq N_j$, which contradicts our choice of $m$ being the largest such integer for which this property holds. Therefore, each quotient group $N_{i+1}/N_i$ is simple, and $1 = N_0 \trianglelefteq N_1 \trianglelefteq ... \trianglelefteq N_m = G$ defines a composition series for $G$.