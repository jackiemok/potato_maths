\subsection*{Exercise 1}
Suppose that $G$ is an abelian simple group. Then, $|G| > 1$, and the only normal subgroups of $G$ are $\{e\}$ and $G$ itself. But since $G$ is abelian, any subgroup $N \le G$ must be normal as well, since $\forall g \in G, n \in N,$ we have that $gng^{-1} = gg^{-1}n = en = n \in N$. Thus, there cannot exist any proper nontrivial subgroups of $G$.

If $|G| = n < \infty$, then $n$ must be prime, or else by Cauchy's Theorem, we could find subgroups of prime orders dividing $n$. This would contradict our choice of $G$ being simple. Otherwise, in the infinite-order case, we could still find a non-trivial subgroup since $G$ is abelian, thereby contradicting our choice of $G$ being simple. This implies that $G \cong \mathbb{Z}_p$, for some prime number $p$, since $p$ has no nontrivial divisors.