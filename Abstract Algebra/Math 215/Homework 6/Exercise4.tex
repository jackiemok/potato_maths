\subsection*{Exercise 1}
Let $R$ be an integral domain with quotient field $F$, and let $p(x)$ be a monic polynomial in $R[x]$, where $p(x) = a(x)b(x)$ for some monic polynomials $a(x),b(x) \in F[x]$ whose sum of degrees is that of $p(x)$. Further suppose that $a(x) \not\in R[x]$. We claim that then $R$ is not a Unique Factorization Domain and thereby deduce that $\mathbb{Z}[2\sqrt{2}]$ is not a Unique Factorization Domain.

Suppose for the sake of contradiction that $R$ is a Unique Factorization Domain. Then, by Gauss' Lemma, there exist some nonzero $r,s \in F$ such that $ra(x) = A(x)$ and $sb(x) = B(x)$, where $A(x),B(x) \in R[x]$. Furthermore, we have $rsa(x)b(x) = ra(x)sb(x) = A(x)B(x) = p(x) \in R[x]$, and since $p(x), a(x), b(x)$ are monic, it follows that $rs = 1$ so that both $r$ and $s$ are units in $F$. Now, since $ra(x) = A(x) \in R[x]$, where both $a(x)$ and $A(x)$ are monic, $r$ is forced to be in $R$. But if $r \in R$, then $a(x) \in R[x]$, contradicting our original assumption that $a(x) \not\in R[x]$. Therefore, $R$ cannot be a Unique Factorization Domain.

Now, we deduce that $\mathbb{Z}[2\sqrt{2}]$ is not a Unique Factorization Domain. Let $p(x) = x^2 - 2 \in \mathbb{Z}[2\sqrt{2}]$, which is monic as previously. Similarly, we write $p(x) = a(x)b(x)$, where $a(x) = (x + \frac{2\sqrt{2}}{2})$ and $b(x) = (x - \frac{2\sqrt{2}}{2})$, which are both monic polynomials in the field of fractions $Q(\mathbb{Z}[2\sqrt{2}])$. As above, notice that $a(x) \not\in \mathbb{Z}[2\sqrt{2}]$, since otherwise we would have $(a + 2b\sqrt{2}) = \sqrt{2}$ for some $a,b \in \mathbb{Z}$, but $2b$ is even while 1 is odd. Therefore, $\mathbb{Z}[2\sqrt{2}]$ is not a Unique Factorization Domain.