\subsection*{Exercise 4}
We claim that the ideals $P_1 = (x)$ and $P_2 = (x,y)$ are prime ideals in $R = \mathbb{Q}[x,y]$, where only the latter ideal is a maximal ideal. Since $\mathbb{Q}$ is an integral domain and is thus commutative, so is $\mathbb{Q}[x,y] = \mathbb{Q}[x][y]$. Now, in view of Propositions 7.4.12 and 7.4.13, we will show that $R/P_1$ is an integral domain and $R/P_2$ is a field. Showing these is equivalent to showing that $P_1$ is prime in $R$ and $P_2$ is maximal in $R$, respectively.

Since $P_1$ and $P_2$ are ideals, by Theorem 7.3.2, $P_1 =$ ker($\phi_1)$ and $P_2 =$ ker($\phi_2)$, where $\phi_1$ and $\phi_2$ are the ring homomorphisms given by the natural projections of $R$ onto $R/P_1$ and $R/P_2$, respectively. Precisely, we define for any $f(x,y) \in R, \phi_1(f(x,y)) = f(0,y) \in \mathbb{Q}$ and $\phi_2(f(x,y)) = f(0,0) \in \mathbb{Q}$. Now, we obtain from the First Isomorphism Theorem for Rings that $R/P_1 \cong \phi_1(R) \{f(0,y)\} = \mathbb{Q}[y]$ and similarly, $R/P_2 \cong \phi_2(R) = \mathbb{Q}$, by surjectivity of $\phi_1$ and $\phi_2$. 

Since $\mathbb{Q}$ is an integral domain, so is $\mathbb{Q}[y] \cong R/P_1 \implies P_1 = (x)$ is a prime ideal in $R$ by Proposition 7.4.13. On the other hand, $\mathbb{Q} \cong R/P_2$ is a field, and thus, $P_2 = (x,y)$ is maximal and prime in $R$ by Proposition 7.4.12 and Corollary 7.4.14. But now since $P_2$ is maximal and $\{x\} \subset \{x,y\} \implies P_1 \subset P_2$, it follows that $P_1$ cannot be maximal in $R$.