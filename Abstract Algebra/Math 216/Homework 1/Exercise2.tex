\subsection*{Exercise 2}
\begin{framed}
\textit{Let M be a module over the integral domain R.}

\textbf{[a]} \textit{Suppose that M has rank n and that $x_1, x_2, ... ,x_n$ is any maximal set of linearly independent elements of M. Let $N = Rx_1 + ... + Rx_n$ be the submodule generated by $x_1, x_2, ... , x_n$. Prove that N is isomorphic to $R^n$ and that the quotient M/N is a torsion R-module (equivalently, the elements $x_1, ..., x_n$ are linearly independent and for any $y \in M$ there is a nonzero element $r \in R$ such that ry can be written as a linear combination $r_1x_1 + ... + r_nx_n$ of the $x_i$).}

\textbf{[b]} \textit{Prove conversely that if M contains a submodule N that is free of rank n (i.e., $N \cong R^n$) such that the quotient M/N is a torsion R-module then M has rank n. [Let $y_1, y_2, ..., y_{n+1}$ be any n + 1 elements of M. Use the fact that M/N is torsion to write $r_iy_i$ as a linear combination of a basis for N for some nonzero elements $r_1, ..., r_{n+1}$ of R. Use an argument as in the proof of Proposition 3 to see that the $r_iy_i$, and hence also the $y_i$, are linearly dependent.]}
\end{framed}

\textbf{[a]} Let $X = \{x_1, ..., x_n\} \subseteq M$ so that $N = RX = XR$ is the submodule of $M$ generated by $X$. Let $R^n$ be the free module of rank $n$ over $R$. First, we show that $N \cong R^n$. By Theorem 10.3.6, there exists a free $R$-module $F(X)$ on the set $X$ satisfying the universal property: if $\phi: X \rightarrow N$ is any map of sets, then there is a unique $R$-module homomorphism $\Phi: F(X) \rightarrow N$ such that $\Phi(x_i) = \phi(x_i)$, for all $i = 1,...,n$. In particular, since $X$ is finite, $F(X) = Rx_1 \oplus ... \oplus Rx_n \cong R^n$. It remains for us to verify that $N \cong F(X)$. But by Proposition 10.3.5, $N = RX = Rx_1 + ... + Rx_n \cong Rx_1 \times ... \times Rx_n$, where $R \cong Rx_i$, for each $i = 1,...,n$ under the map $r \mapsto rx_i$. Therefore, $N \cong F(X) \cong R^n$, as desired.

Next, we show that Tor($M/N) = M/N$ (i.e., $M/N$ is a torsion $R$-module by Exercise 10.3.4). Consider the natural projection map $\pi: M \rightarrow M/N$ defined by $\pi(x) = x + N$ as an $R$-module homomorphism with kernel $N$, as in Proposition 10.2.3. By Exercise 10.2.8, $\pi($Tor($M)) \subseteq$ Tor($M/N) \subseteq M/N$. But notice that $\pi($Tor($M)) =$ Tor($M) + N =$ Tor($M) + RX = \{x \in M \mid rx = 0$, for some nonzero $r \in R\} + RX = M/N$. To see this, let $x \in$ Tor($M)$ and $y = \sum_{i=1}^n r_ix_i \in N = RX$, for unique $r_i \in R$, so $(x + y) = x + \sum_{i=1}^nr_ix_i$. If $r \not= 0 \in$ Ann($\{x\})$, then $r(x + y) = rx + ry = 0 + r\sum_{i=1}^nr_ix_i \in N$. Thus, Tor($M/N) = M/N.$

\textbf{[b]} Conversely, suppose that $N$ is a free submodule of $M$ of rank $n$ ($N \cong R^n$) such that Tor($M/N) = M/N$. Suppose that $N$ is free on $X = \{x_1,...,x_n\}$, a subset of $n$ $R$-linearly independent elements. This implies that the rank (maximum number of $R$-linearly independent elements) of $M$ is at least $n$.

Let $Y = \{y_1,...,y_{n+1}\} \subseteq M$. Considering again the map $\pi: M \rightarrow M/N$ defined by $\pi(x) = x + N$, where $M/N =$ Tor($M/N$), we see that $\pi(y_i) = y_i + N$, for each $i = 1,...,n+1$. $Y + N \in M/N$ has a nonzero annihilator, by Exercise 10.3.4. Thus, for each $i = 1,...,n+1$, there exists some nonzero $r \in R$ with $ry_i \in N$. Since by assumption $N$ is free of rank $n$, by Proposition 12.1.3, there exists an $R$-linear dependence between the $ry_i \in N$, for $i = 1,...,n+1$. Therefore, $M$ cannot have rank greater than $n$, so $M$ has rank $n$.