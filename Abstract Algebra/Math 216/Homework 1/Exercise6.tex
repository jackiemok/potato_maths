\subsection*{Exercise 8}
\begin{framed}
\textit{Let R be a P.I.D., let B be a torsion R-module and let p be a prime in R. Prove that if pb = 0 for some nonzero $b \in B$, then Ann($B) \subseteq (p)$.}
\end{framed}

Since $R$ is a P.I.D. and thus, an integral domain, it has no zero divisors. If $pb = 0$, for some prime $p \in R$ and nonzero $b \in B =$ Tor($B)$, then $p$ annihilates $b$ so that $p \in$ Ann($\{b\}) = \{r \in R \mid rb = 0\}$. But since $p$ is prime and the annihilator of $\{b\}$ is an ideal of the P.I.D. $R$, then $(p) =$ Ann($\{b\})$ = $R$(Ann($\{b\})$) = Ann($R\{b\})$ is maximal. $R\{b\} \subseteq B$ is the smallest submodule of $B$ containing $\{b\}$ by the Submodule Criterion, so by the definition of the annihilator of a submodule (page 460), Ann($B) \subseteq$ Ann($R\{b\}) = (p)$.