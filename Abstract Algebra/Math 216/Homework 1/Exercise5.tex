\subsection*{Exercise 5}
\begin{framed}
\textit{Let $R = \mathbb{Z}[x]$ and let M = $(2,x)$ be the ideal generated by $2$ and x, considered as a submodule of R. Show that $\{2,x\}$ is not a basis of M. [Find a nontrivial R-linear dependence between these two elements.] Show that the rank of M is 1 but that M is not free of rank 1 (cf. Exercise $2$).}
\end{framed}
Suppose for the sake of contradiction that $\{2,x\}$ is indeed a basis for the $\mathbb{Z}[x]$-submodule $M = (2,x)$. Then, each $f(x) \in M$ could be represented as $f(x) = r_1(x)x + r_2(x)2$, for unique $r_1(x), r_2(x) \in \mathbb{Z}[x]$. But consider the case when $f(x) = 0 \in M$. Letting $r_1(x) = -2$ and $r_2(x) = x$ yields $f(x) = -2x + 2x = 0$, contradicting the linear independence requirement that $r_1(x) = 0 = r_2(x)$, so $\{2,x\}$ cannot be a basis for $M$.

Now, we must show that $M$ is not free but has rank 1. Notice from the construction above that given any other basis of size two, the same problem arises. Namely, if $f(x) \in M$ and $\{m_1(x), m_2(x)\} \subseteq M$ is a basis for $M$ so that $f(x) = r_1(x)m_1(x) + r_2(x)m_2(x)$, for unique $r_1(x), r_2(x) \in \mathbb{Z}[x]$, then letting $r_1(x) = -m_2(x) \not= 0$ and $r_2(x) = m_1(x) \not= 0$ would give $f(x) = -m_2(x)m_1(x) + m_2(x)m_1(x) = 0$. Therefore, the rank (maximum number of $R$-linearly independent elements) of $M$ must be either 0 or 1. By Theorem 12.1.5 and Exercise 12.1.1, $M$ has rank 0 if and only if $M$ = Tor($M$). But since $M$ is an ideal of the integral domain $R = \mathbb{Z}[x]$, it has no zero divisors and thus, $M \not=$ Tor($M)$. We conclude that the rank of $M$ is 1.

Suppose to the contrary that $M$ is free of rank 1 on the set $A = \{a(x)\}$. Since by assumption $M = (2,x)$, $RA = M$ must contain both 2 and $x$. But deg(2) = 0 $\not= 1 =$ deg($x)$, so $r(x)a(x) = 2$, for unique $r(x) \in R \implies$ deg($r(x)) = 0 =$ deg($a(x))$. Meanwhile, $s(x)a(x) = x$, for unique $s(x) \in R$, forces deg($s(x)) = 1$. Then, $a(x)$ is a constant, so $a(x) =: a \in \mathbb{Z}$ divides both 2 and $x$ in $M$. But $x$ is monic and $s(x)$ must have coefficients in $\mathbb{Z}$, so $a$ cannot divide $x$. Therefore, $A$ is not a basis for $M$, and $M$ is not free of rank 1.